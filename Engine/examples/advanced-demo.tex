    \documentclass{article}
     \usepackage{amsmath, amssymb, mathtools, bm, physics,
     tensor, nicematrix}
     \usepackage{algorithm}
     \usepackage{algorithmic}

     \title{Advanced Mathematical Typesetting Showcase}
     \author{LaTeX to HTML Engine}
     \date{\today}

     \begin{document}

     \maketitle

     \section{Mathematical Typesetting}

     \subsection{Basic Mathematical Expressions}

     The famous quadratic formula:
     \[
     x = \frac{-b \pm \sqrt{b^2 - 4ac}}{2a}
     \]

     Pythagorean theorem:
     \[
     a^2 + b^2 = c^2
     \]

     Complex numbers:
     \[
     z = a + bi = re^{i\theta} = r(\cos\theta +
     i\sin\theta)
     \]

     \subsection{Calculus and Analysis}

     Fundamental theorem of calculus:
     \[
     \int_{a}^{b} f'(x) dx = f(b) - f(a)
     \]

     Taylor series expansion:
     \[
     f(x) = \sum_{n=0}^{\infty}
     \frac{f^{(n)}(a)}{n!}(x-a)^n
     \]

     Limit definition:
     \[
     \lim_{h \to 0} \frac{f(x+h) - f(x)}{h} = f'(x)
     \]

     Multiple integrals:
     \[
     \iiint_{V} f(x,y,z) \, dV =
     \int_{a}^{b}\int_{c}^{d}\int_{e}^{f} f(x,y,z) \, dz \,
      dy \, dx
     \]

     \subsection{Linear Algebra}

     Matrix multiplication:
     \[
     \begin{pmatrix}
     a_{11} & a_{12} & a_{13} \\
     a_{21} & a_{22} & a_{23} \\
     a_{31} & a_{32} & a_{33}
     \end{pmatrix}
     \begin{pmatrix}
     x_1 \\
     x_2 \\
     x_3
     \end{pmatrix}
     =
     \begin{pmatrix}
     b_1 \\
     b_2 \\
     b_3
     \end{pmatrix}
     \]

     Determinant of 3×3 matrix:
     \[
     \det(A) = \begin{vmatrix}
     a & b & c \\
     d & e & f \\
     g & h & i
     \end{vmatrix} = a(ei - fh) - b(di - fg) + c(dh - eg)
     \]

     Eigenvalue equation: $\mathbf{A}\mathbf{v} =
     \lambda\mathbf{v}$

     \section{Physics Notation}

     \subsection{Classical Mechanics}

     Newton's second law in vector form:
     \[
     \mathbf{F} = m\mathbf{a} = m\frac{d\mathbf{v}}{dt} =
     m\frac{d^2\mathbf{r}}{dt^2}
     \]

     Kinetic and potential energy:
     \[
     E = K + U = \frac{1}{2}mv^2 + mgh
     \]

     Lagrangian formulation:
     \[
     L = T - V \quad \text{and} \quad
     \frac{d}{dt}\left(\frac{\partial L}{\partial
     \dot{q}_i}\right) - \frac{\partial L}{\partial q_i} =
     0
     \]

     \subsection{Electromagnetism}

     Maxwell's equations in differential form:
     \begin{align}
     \nabla \cdot \mathbf{E} &= \frac{\rho}{\epsilon_0} \\
     \nabla \cdot \mathbf{B} &= 0 \\
     \nabla \times \mathbf{E} &= -\frac{\partial
     \mathbf{B}}{\partial t} \\
     \nabla \times \mathbf{B} &= \mu_0 \mathbf{J} +
     \mu_0\epsilon_0 \frac{\partial \mathbf{E}}{\partial t}
     \end{align}

     Lorentz force law:
     \[
     \mathbf{F} = q(\mathbf{E} + \mathbf{v} \times
     \mathbf{B})
     \]

     \subsection{Quantum Mechanics}

     Schrödinger equation:
     \[
     i\hbar\frac{\partial\psi}{\partial t} = \hat{H}\psi
     \]

     Momentum operator:
     \[
     \hat{p} = -i\hbar\nabla
     \]

     Hamiltonian for harmonic oscillator:
     \[
     \hat{H} = \frac{\hat{p}^2}{2m} +
     \frac{1}{2}m\omega^2\hat{x}^2
     \]

     Physics operators using physics package:
     \begin{align}
     \expval{H} &= \bra{\psi} \hat{H} \ket{\psi} \\
     \comm{x}{p} &= i\hbar \\
     \anticomm{a}{a^\dagger} &= 1
     \end{align}

     \section{Tensor Mathematics}

     \subsection{Basic Tensor Operations}

     Metric tensor contraction:
     \[
     g_{\mu\nu}g^{\nu\rho} = \delta_\mu^\rho
     \]

     Christoffel symbols:
     \[
     \Gamma^\lambda_{\mu\nu} =
     \frac{1}{2}g^{\lambda\rho}\left(\frac{\partial
     g_{\rho\mu}}{\partial x^\nu} + \frac{\partial
     g_{\rho\nu}}{\partial x^\mu} - \frac{\partial
     g_{\mu\nu}}{\partial x^\rho}\right)
     \]

     Riemann curvature tensor:
     \[
     R^\rho_{\ \sigma\mu\nu} = \frac{\partial
     \Gamma^\rho_{\nu\sigma}}{\partial x^\mu} -
     \frac{\partial \Gamma^\rho_{\mu\sigma}}{\partial
     x^\nu} +
     \Gamma^\rho_{\mu\lambda}\Gamma^\lambda_{\nu\sigma} -
     \Gamma^\rho_{\nu\lambda}\Gamma^\lambda_{\mu\sigma}
     \]

     \subsection{Relativity}

     Einstein field equations:
     \[
     G_{\mu\nu} + \Lambda g_{\mu\nu} = \frac{8\pi
     G}{c^4}T_{\mu\nu}
     \]

     Minkowski metric:
     \[
     ds^2 = -c^2dt^2 + dx^2 + dy^2 + dz^2 =
     \eta_{\mu\nu}dx^\mu dx^\nu
     \]

     Four-momentum:
     \[
     p^\mu = (E/c, p_x, p_y, p_z)
     \]

     \section{Advanced Matrices}

     \subsection{Special Matrix Types}

     Identity matrix and its properties:
     \[
     I_n = \begin{pmatrix}
     1 & 0 & \cdots & 0 \\
     0 & 1 & \cdots & 0 \\
     \vdots & \vdots & \ddots & \vdots \\
     0 & 0 & \cdots & 1
     \end{pmatrix}, \quad AI_n = I_nA = A
     \]

     Pauli matrices:
     \begin{align}
     \sigma_1 &= \begin{pmatrix} 0 & 1 \\ 1 & 0
     \end{pmatrix} \\
     \sigma_2 &= \begin{pmatrix} 0 & -i \\ i & 0
     \end{pmatrix} \\
     \sigma_3 &= \begin{pmatrix} 1 & 0 \\ 0 & -1
     \end{pmatrix}
     \end{align}

     \subsection{NiceMatrix Examples}

     Augmented matrix for linear system:
     \[
     \left[
     \begin{pNiceArray}{ccc|c}
     2 & 3 & 1 & 8 \\
     1 & -1 & 2 & 3 \\
     3 & 2 & -1 & 1
     \end{pNiceArray}
     \right]
     \]

     Block matrix:
     \[
     \begin{pNiceMatrix}
     A & B \\
     C & D
     \end{pNiceMatrix}
     =
     \begin{pmatrix}
     a_{11} & \cdots & a_{1k} & b_{11} & \cdots & b_{1m} \\
     \vdots & \ddots & \vdots & \vdots & \ddots & \vdots \\
     a_{k1} & \cdots & a_{kk} & b_{k1} & \cdots & b_{km} \\
     c_{11} & \cdots & c_{1k} & d_{11} & \cdots & d_{1m} \\
     \vdots & \ddots & \vdots & \vdots & \ddots & \vdots \\
     c_{n1} & \cdots & c_{nk} & d_{n1} & \cdots & d_{nm}
     \end{pmatrix}
     \]

     \section{Algorithms}

     \subsection{Numerical Methods}

     \begin{algorithm}[h]
     \caption{Gradient Descent Optimization}
     \label{algo:gradient}
     \begin{algorithmic}
     \State \Input: Function $f(x)$, learning rate
     $\alpha$, tolerance $\epsilon$
     \State \Output: Local minimum $x^*$
     \State $x \gets x_0$ (initial guess)
     \While{$\|\nabla f(x)\| > \epsilon$}
         \State $g \gets \nabla f(x)$ (compute gradient)
         \State $x \gets x - \alpha g$ (update parameters)
         \State $f_{new} \gets f(x)$
         \If{$|f_{new} - f_{old}| < \epsilon$}
             \State \Return $x$
         \EndIf
         \State $f_{old} \gets f_{new}$
     \EndWhile
     \State \Return $x$
     \end{algorithmic}
     \end{algorithm}

     \subsection{Matrix Operations}

     \begin{algorithm}[h]
     \caption{Gaussian Elimination}
     \label{algo:gaussian}
     \begin{algorithmic}
     \State \Input: Matrix $A \in \mathbb{R}^{n \times n}$,
      vector $b \in \mathbb{R}^n$
     \State \Output: Solution vector $x$ to $Ax = b$
     \State Create augmented matrix $[A|b]$
     \For{$i = 1$ to $n-1$}
         \For{$j = i+1$ to $n$}
             \State $factor \gets A[j,i]/A[i,i]$
             \For{$k = i$ to $n$}
                 \State $A[j,k] \gets A[j,k] - factor \cdot
      A[i,k]$
             \EndFor
             \State $b[j] \gets b[j] - factor \cdot b[i]$
         \EndFor
     \EndFor
     \State \Return Back substitution to find $x$
     \end{algorithmic}
     \end{algorithm}

     \begin{algorithm}[h]
     \caption{Matrix Multiplication}
     \label{algo:matrix}
     \begin{algorithmic}
     \State \Input: Matrices $A \in \mathbb{R}^{m \times
     p}$, $B \in \mathbb{R}^{p \times n}$
     \State \Output: Matrix $C = AB \in \mathbb{R}^{m
     \times n}$
     \For{$i = 1$ to $m$}
         \For{$j = 1$ to $n$}
             \State $C[i,j] \gets 0$
             \For{$k = 1$ to $p$}
                 \State $C[i,j] \gets C[i,j] + A[i,k] \cdot
      B[k,j]$
             \EndFor
         \EndFor
     \EndFor
     \State \Return $C$
     \end{algorithmic}
     \end{algorithm}

     \section{Tables}

     \subsection{Mathematical Constants}

     \begin{table}[h]
     \centering
     \caption{Important Mathematical Constants}
     \label{tab:constants}
     \begin{tabular}{|l|c|c|l|}
     \hline
     \textbf{Constant} & \textbf{Symbol} & \textbf{Value} &
      \textbf{Description} \\
     \hline
     Pi & $\pi$ & 3.14159265359 & Circle circumference
     ratio \\
     Euler's number & $e$ & 2.71828182846 & Base of natural
      logarithm \\
     Golden ratio & $\phi$ & 1.61803398875 & Aesthetic
     proportion \\
     Euler-Mascheroni & $\gamma$ & 0.57721566490 & Limit of
      harmonic series \\
     \hline
     \end{tabular}
     \end{table}

     \subsection{Linear Algebra Operations}

     \begin{table}[h]
     \centering
     \caption{Matrix Operations Complexity}
     \label{tab:complexity}
     \begin{tabular}{|l|c|c|c|}
     \hline
     \textbf{Operation} & \textbf{Dimensions} &
     \textbf{Time Complexity} & \textbf{Space Complexity}
     \\
     \hline
     Matrix Addition & $m \times n$ & $O(mn)$ & $O(mn)$ \\
     Matrix Multiplication & $m \times p$, $p \times n$ &
     $O(mnp)$ & $O(mn)$ \\
     Determinant (LU) & $n \times n$ & $O(n^3)$ & $O(n^2)$
     \\
     Inverse (Gaussian) & $n \times n$ & $O(n^3)$ &
     $O(n^2)$ \\
     Eigenvalues (QR) & $n \times n$ & $O(n^3)$ & $O(n^2)$
     \\
     \hline
     \end{tabular}
     \end{table}

     \subsection{Integration Methods}

     \begin{table}[h]
     \centering
     \caption{Numerical Integration Methods Comparison}
     \label{tab:integration}
     \begin{tabular}{|l|c|c|c|c|}
     \hline
     \textbf{Method} & \textbf{Order} & \textbf{Error Term}
      & \textbf{Stability} & \textbf{Applications} \\
     \hline
     Rectangle Rule & 1 & $O(h^2)$ & Poor & Simple
     integrals \\
     Trapezoidal Rule & 2 & $O(h^3)$ & Fair & Smooth
     functions \\
     Simpson's Rule & 4 & $O(h^5)$ & Good & Accurate
     integration \\
     Gaussian Quadrature & $2n-1$ & $O(h^{2n})$ & Excellent
      & High precision \\
     \hline
     \end{tabular}
     \end{table}

     \section{Advanced Mathematical Topics}

     \subsection{Fourier Series}

     Fourier series representation:
     \[
     f(x) = \frac{a_0}{2} + \sum_{n=1}^{\infty} \left(
     a_n\cos\frac{n\pi x}{L} + b_n\sin\frac{n\pi x}{L}
     \right)
     \]

     Fourier coefficients:
     \[
     a_n = \frac{1}{L}\int_{-L}^{L} f(x)\cos\frac{n\pi
     x}{L}\,dx, \quad b_n = \frac{1}{L}\int_{-L}^{L}
     f(x)\sin\frac{n\pi x}{L}\,dx
     \]

     \subsection{Complex Analysis}

     Cauchy's integral formula:
     \[
     f^{(n)}(a) = \frac{n!}{2\pi i}\oint_C
     \frac{f(z)}{(z-a)^{n+1}}dz
     \]

     Residue theorem:
     \[
     \oint_C f(z)dz = 2\pi i \sum_{k} \text{Res}(f, z_k)
     \]
     
     \subsection{Differential Equations}

     Second-order linear ODE:
     \[
     y'' + p(x)y' + q(x)y = r(x)
     \]

     Solution using variation of parameters:
     \[
     y(x) = C_1y_1(x) + C_2y_2(x) -
     y_1(x)\int\frac{y_2(x)r(x)}{W(x)}dx +
     y_2(x)\int\frac{y_1(x)r(x)}{W(x)}dx
     \]

     where $W(x) = y_1y_2' - y_2y_1'$ is the Wronskian.

     \section{Conclusion}

     This document demonstrates the comprehensive
     mathematical typesetting capabilities of the LaTeX to
     HTML engine, including:

     \begin{itemize}
     \item Basic mathematical expressions and calculus
     \item Physics notation with classical mechanics,
     electromagnetism, and quantum mechanics
     \item Tensor mathematics for relativity and
     differential geometry
     \item Advanced matrix operations and linear algebra
     \item Algorithmic pseudocode for numerical methods
     \item Professional tables for data presentation
     \item Complex analysis and differential equations
     \end{itemize}

     All mathematical notation is properly converted to
     high-quality HTML with appropriate styling and
     formatting.

     \end{document}
