\documentclass{portfolio}
\usepackage[utf8]{inputenc}

\title{Dr. Jane Smith}
\author{Machine Learning Researcher}
\date{2024}

\begin{document}

\maketitle

% Custom navigation logo (optional)
\navlogo{profile_photo.png}

% Custom navigation links (optional)
\nabbar{Research, Publications, Blog, Contact}

% Custom sidebar content with profile image (optional)
% Circle shape (default)
\sidebar{\profileimage[shape=circle]{profile_photo.png}Dr. Jane Smith | Senior Research Scientist | MIT, Cambridge, MA}

% Uncomment below to try different shapes:
% Rectangle shape:
% \sidebar{\profileimage[shape=rectangle]{profile_photo.png}Dr. Jane Smith | Senior Research Scientist | MIT, Cambridge, MA}
% Default shape (rounded square):
% \sidebar{\profileimage[shape=default]{profile_photo.png}Dr. Jane Smith | Senior Research Scientist | MIT, Cambridge, MA}

% Additional LaTeX content for the portfolio
\section{Research Interests}

I work on \textbf{deep learning} and \textbf{reinforcement learning} with applications to robotics and healthcare.

\subsection{Current Projects}

\begin{itemize}
    \item Neural architecture search for computer vision
    \item Federated learning for medical imaging
    \item Autonomous navigation algorithms
\end{itemize}

\section{Featured Work}

Here are some examples of my recent work:

% Image in main content with width 80% of text width
\includegraphics[width=0.8\textwidth]{profile_photo.png}

\subsection{Research Images}

% Simple image without options (default size)
\includegraphics{profile_photo.png}

% Image with specific width in pixels
\includegraphics[width=200px]{profile_photo.png}

% Image with both width and height
\includegraphics[width=150px,height=100px]{profile_photo.png}

% Image with figure environment
\begin{figure}
\includegraphics[width=5cm]{profile_photo.png}
\caption{Research visualization showing deep learning model performance}
\end{figure}

\section{Publications and Data}

\subsection{Recent Publications Table}

\begin{table}[h]
\centering
\begin{tabular}{|l|c|p{6cm}|}
\hline
\textbf{Year} & \textbf{Venue} & \textbf{Title} \\
\hline
2024 & ICML & Deep Learning for Scientific Discovery \\
\hline
2024 & NeurIPS & Efficient Training of Large Language Models \\
\hline
2023 & CVPR & Neural Architecture Search for Computer Vision \\
\hline
\end{tabular}
\caption{Selected publications from 2023-2024}
\end{table}

\subsection{Mathematical Framework}

Our approach is based on the following optimization equation:

\begin{equation}
\min_{\theta} \mathcal{L}(\theta) = \frac{1}{N} \sum_{i=1}^{N} \ell(f(x_i; \theta), y_i) + \lambda \|\theta\|_2^2
\end{equation}

Where we use the cross-entropy loss:

\begin{align}
\ell(f(x_i; \theta), y_i) &= -\sum_{c=1}^{C} y_{i,c} \log(\frac{e^{f_c(x_i; \theta)}}{\sum_{k=1}^{C} e^{f_k(x_i; \theta)}}) \\
&= -\log(\frac{e^{f_{y_i}(x_i; \theta)}}{\sum_{k=1}^{C} e^{f_k(x_i; \theta)}})
\end{align}

\subsection{Research Areas}

\begin{columns}[T]
\begin{column}{0.5\textwidth}
\textbf{Machine Learning}
\begin{itemize}
\item Deep neural networks
\item Reinforcement learning
\item Transfer learning
\end{itemize}
\end{column}
\begin{column}{0.5\textwidth}
\textbf{Applications}
\begin{itemize}
\item Computer vision
\item Natural language processing
\item Healthcare systems
\end{itemize}
\end{column}
\end{columns}

\section{Education}

\textbf{Ph.D. in Computer Science} - Stanford University (2020)

\textbf{M.S. in Mathematics} - MIT (2018)

\textbf{B.S. in Physics} - UC Berkeley (2016)

\end{document}