% Professional eBook Example with Security Features
% @screenshot:disable
% @copytext:disable
% @print:disable

\documentclass{book}

\title{Advanced Web Development}
\author{John Doe}
\date{\today}
\publisher{Professional Publishing House}
\location{New York, USA}
\isbn{978-0-123456-78-9}
\edition{Second}

\begin{document}

\frontmatter

\begin{titlepage}
\title{Advanced Web Development}
\author{John Doe}
\date{October 2024}
\publisher{Professional Publishing House}
\location{New York, USA}
\isbn{978-0-123456-78-9}
\edition{Second}
\end{titlepage}

\begin{dedication}
Dedicated to my family and all the aspiring developers who dream of creating something extraordinary.
\end{dedication}

\begin{thanks}
Special thanks to the open-source community and everyone who contributed to making web development accessible and enjoyable.
\end{thanks}

\tableofcontents

\mainmatter

\part{Fundamentals}

\chapter{Introduction to Modern Web Development}
\label{chap:intro}

Web development has evolved dramatically over the past decade. From simple static pages to complex, interactive applications, the landscape continues to transform with new technologies, frameworks, and paradigms emerging regularly.

\section{The Evolution of Web Technologies}
The journey began with HTML 1.0, a simple markup language designed for basic document structure. Today, we have HTML5, CSS3, and ECMAScript 2023, providing developers with powerful tools for creating rich, interactive experiences.

\subsection{From Static to Dynamic}
Early websites were essentially digital brochures -- static documents with limited interactivity. The introduction of JavaScript transformed the web into a dynamic platform capable of handling complex applications and real-time interactions.

\subsection{The Rise of Frameworks}
The complexity of modern web applications led to the development of frameworks like React, Vue, and Angular, which provide structured approaches to building scalable, maintainable applications.

\chapter{Core Concepts and Principles}
\label{chap:concepts}

\section{Understanding the Web Stack}
Modern web development involves understanding multiple layers of technology, from the browser's rendering engine to server-side processing and database management.

\subsection{Frontend Technologies}
Frontend development focuses on user interface and user experience. Key technologies include:
\begin{itemize}
    \item HTML5 for structure
    \item CSS3 for styling and layout
    \item JavaScript for interactivity
    \item Responsive design principles
\end{itemize}

\subsection{Backend Technologies}
Backend development handles server-side logic, data processing, and application architecture. Common technologies include Node.js, Python, Ruby, and various database systems.

\begin{theorem}[Fundamental Principle]
The separation of concerns is a fundamental principle in software engineering that promotes modular, maintainable code by separating distinct functionalities into independent modules.
\end{theorem}

\begin{proof}
By separating concerns, we can modify one aspect of the system without affecting others, making the codebase more maintainable and scalable. Each module can be developed, tested, and deployed independently.
\end{proof}

\chapter{Development Best Practices}
\label{chap:practices}

\section{Writing Clean Code}
Clean code is essential for long-term maintainability and collaboration. It follows clear conventions, is well-documented, and avoids unnecessary complexity.

\subsection{Code Organization}
Organize code logically with clear directory structures, meaningful file names, and consistent naming conventions. This makes the codebase easier to navigate and understand.

\subsection{Documentation Practices}
Good documentation serves as a roadmap for developers, explaining the architecture, design decisions, and usage patterns. It should be clear, concise, and kept up-to-date.

\part{Advanced Topics}

\chapter{Performance Optimization}
\label{chap:performance}

Performance optimization is crucial for user experience and search engine rankings. It involves various techniques to make web applications faster and more efficient.

\section{Frontend Optimization}
Frontend optimization focuses on reducing load times and improving rendering performance.

\subsection{Image Optimization}
Images often account for the majority of a web page's size. Techniques include:
\begin{itemize}
    \item Compressing images without significant quality loss
    \item Using appropriate image formats (WebP, AVIF)
    \item Implementing lazy loading for off-screen images
    \item Using responsive images with srcset
\end{itemize}

\subsection{JavaScript Performance}
Optimizing JavaScript involves minimizing parse time, reducing memory usage, and improving execution speed.

\chapter{Security Considerations}
\label{chap:security}

Web security is a critical aspect of development that must be addressed throughout the entire development lifecycle.

\section{Common Vulnerabilities}
Understanding common vulnerabilities helps developers write more secure code.

\subsection{Cross-Site Scripting (XSS)}
XSS attacks occur when malicious scripts are injected into web applications, potentially compromising user data and session information.

\subsection{SQL Injection}
SQL injection attacks target database-driven applications by manipulating SQL queries to access or modify unauthorized data.

\begin{theorem}[Security Principle]
The principle of least privilege states that users and applications should only have the minimum level of access necessary to perform their functions.
\end{theorem}

\appendix

\chapter{Additional Resources}
\label{chap:resources}

\section{Recommended Reading}
\begin{itemize}
    \item \emph{Clean Code} by Robert C. Martin
    \item \emph{Design Patterns} by the Gang of Four
    \item \emph{The Pragmatic Programmer} by Andrew Hunt and David Thomas
\end{itemize}

\section{Online Communities}
\begin{itemize}
    \item Stack Overflow
    \item GitHub
    \item Developer forums and communities
    \item Open source projects
\end{itemize}

\end{document}